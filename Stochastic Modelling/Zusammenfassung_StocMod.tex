\documentclass[11pt]{article}

\usepackage{comment} % enables the use of multi-line comments (\ifx \fi) 
\usepackage[a4paper,margin=1cm]{geometry}
\usepackage[utf8]{inputenc}
\usepackage[ngerman]{isodate}
\usepackage{gensymb}
\usepackage{graphicx}
\usepackage{booktabs}% http://ctan.org/pkg/booktabs
\usepackage{tabularx}
\usepackage{ltablex} % Longtables with tabularx
\usepackage[x11names]{xcolor}
\usepackage{amsmath}
\usepackage{amssymb}
\usepackage{array}
\usepackage{wrapfig}
\usepackage{subcaption}
\usepackage{csquotes}
\usepackage{lscape}
\usepackage{geometry}
\usepackage{multicol}
\usepackage{bm}
\usepackage{enumitem}
\usepackage{hyperref}
\usepackage{mdframed}
\usepackage{scalerel}
\usepackage{stackengine}
\usepackage{mathtools}
\usepackage{pdfpages}

% Code highlighting
\usepackage{minted}
\surroundwithmdframed{minted}

% Be able to caption equations and float them in place
\usepackage{float}

\newmdtheoremenv{theorem}{Theorem}
\geometry{a4paper, margin=2.4cm}

\newcommand\equalhat{\mathrel{\stackon[1.5pt]{=}{\stretchto{\scalerel*[\widthof{=}]{\wedge}{\rule{1ex}{3ex}}}{0.5ex}}}}
\newcommand\defeq{\mathrel{\overset{\makebox[0pt]{\mbox{\normalfont\tiny def}}}{=}}}
\newcolumntype{C}{>{\centering\arraybackslash}X}

\DeclarePairedDelimiter\abs{\lvert}{\rvert}
\DeclarePairedDelimiter\norm{\lVert}{\rVert}

\setcounter{tocdepth}{3}
\setcounter{secnumdepth}{3}

\graphicspath{{./img/}}

\begin{document}
	
\title{Stochastic Modelling FS20}
\author{Pascal Baumann\\pascal.baumann@stud.hslu.ch}
\maketitle



For errors or improvement raise an issue or make a pull request on the \href{https://github.com/KilnOfTheSecondFlame/mse_summaries}{github repository}.

\tableofcontents
\newpage



\section{Introduction}

Return Value Stock Market
\begin{equation*}
	\log_{10}\left(\frac{S_t}{S_{t-1}}\right)
\end{equation*}

Despite the variety of random sources there is structure in the noise. The goal is to make meaningful predictions from these structures.

\begin{theorem}
	Given the set of events $\Omega$
	\begin{enumerate}[label=\Roman*.]
		\item $P: \{\text{Events in }\Omega\}\rightarrow [0,1]$ such that $\text{E} \mapsto P(\text{E})\in[0,1]$
		\item $P(\Omega) = 1$
		\item For every sequence of mutually incompatible Events $E_i$ ($E_i\cap E_j$ if $i\neq j$)
		\begin{equation*}
			P\left(\bigcup_{i=1}^\infty E_i\right) = \sum_{i=1}^{\infty} P(E_i)
		\end{equation*}
	\end{enumerate}
	$P(\text E)$ is referred to as the probability of the event $E$.
\end{theorem}

\subsection{Conditional Probability}
\begin{theorem}
	The conditional probability of the event $E$ given $F$
	\begin{equation*}
		P\left( E \middle| F \right) = \frac{P(E\cap F)}{P(F)}
	\end{equation*}
\end{theorem}

\begin{theorem}
	Bayes' rule is given by
	\begin{equation*}
		P(E|F) = \frac{P(F|E)\cdot P(E)}{P(F)}
	\end{equation*}
\end{theorem}

If $A$ and $B$ are incompatible then $A\cap B = \emptyset$.
\begin{theorem}
	Two events $A$ and $B$ are independent if
	\begin{equation*}
		P(A\cup B) = P(A)\cdot P(B)
	\end{equation*}
\end{theorem}


\end{document}
